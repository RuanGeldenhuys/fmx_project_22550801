\documentclass[11pt,preprint, authoryear]{elsarticle}

\usepackage{lmodern}
%%%% My spacing
\usepackage{setspace}
\setstretch{1.2}
\DeclareMathSizes{12}{14}{10}{10}

% Wrap around which gives all figures included the [H] command, or places it "here". This can be tedious to code in Rmarkdown.
\usepackage{float}
\let\origfigure\figure
\let\endorigfigure\endfigure
\renewenvironment{figure}[1][2] {
    \expandafter\origfigure\expandafter[H]
} {
    \endorigfigure
}

\let\origtable\table
\let\endorigtable\endtable
\renewenvironment{table}[1][2] {
    \expandafter\origtable\expandafter[H]
} {
    \endorigtable
}


\usepackage{ifxetex,ifluatex}
\usepackage{fixltx2e} % provides \textsubscript
\ifnum 0\ifxetex 1\fi\ifluatex 1\fi=0 % if pdftex
  \usepackage[T1]{fontenc}
  \usepackage[utf8]{inputenc}
\else % if luatex or xelatex
  \ifxetex
    \usepackage{mathspec}
    \usepackage{xltxtra,xunicode}
  \else
    \usepackage{fontspec}
  \fi
  \defaultfontfeatures{Mapping=tex-text,Scale=MatchLowercase}
  \newcommand{\euro}{€}
\fi

\usepackage{amssymb, amsmath, amsthm, amsfonts}

\def\bibsection{\section*{References}} %%% Make "References" appear before bibliography


\usepackage[round]{natbib}

\usepackage{longtable}
\usepackage[margin=2.3cm,bottom=2cm,top=2.5cm, includefoot]{geometry}
\usepackage{fancyhdr}
\usepackage[bottom, hang, flushmargin]{footmisc}
\usepackage{graphicx}
\numberwithin{equation}{section}
\numberwithin{figure}{section}
\numberwithin{table}{section}
\setlength{\parindent}{0cm}
\setlength{\parskip}{1.3ex plus 0.5ex minus 0.3ex}
\usepackage{textcomp}
\renewcommand{\headrulewidth}{0.2pt}
\renewcommand{\footrulewidth}{0.3pt}

\usepackage{array}
\newcolumntype{x}[1]{>{\centering\arraybackslash\hspace{0pt}}p{#1}}

%%%%  Remove the "preprint submitted to" part. Don't worry about this either, it just looks better without it:
\makeatletter
\def\ps@pprintTitle{%
  \let\@oddhead\@empty
  \let\@evenhead\@empty
  \let\@oddfoot\@empty
  \let\@evenfoot\@oddfoot
}
\makeatother

 \def\tightlist{} % This allows for subbullets!

\usepackage{hyperref}
\hypersetup{breaklinks=true,
            bookmarks=true,
            colorlinks=true,
            citecolor=blue,
            urlcolor=blue,
            linkcolor=blue,
            pdfborder={0 0 0}}


% The following packages allow huxtable to work:
\usepackage{siunitx}
\usepackage{multirow}
\usepackage{hhline}
\usepackage{calc}
\usepackage{tabularx}
\usepackage{booktabs}
\usepackage{caption}


\newenvironment{columns}[1][]{}{}

\newenvironment{column}[1]{\begin{minipage}{#1}\ignorespaces}{%
\end{minipage}
\ifhmode\unskip\fi
\aftergroup\useignorespacesandallpars}

\def\useignorespacesandallpars#1\ignorespaces\fi{%
#1\fi\ignorespacesandallpars}

\makeatletter
\def\ignorespacesandallpars{%
  \@ifnextchar\par
    {\expandafter\ignorespacesandallpars\@gobble}%
    {}%
}
\makeatother

\newenvironment{CSLReferences}[2]{%
}

\urlstyle{same}  % don't use monospace font for urls
\setlength{\parindent}{0pt}
\setlength{\parskip}{6pt plus 2pt minus 1pt}
\setlength{\emergencystretch}{3em}  % prevent overfull lines
\setcounter{secnumdepth}{5}

%%% Use protect on footnotes to avoid problems with footnotes in titles
\let\rmarkdownfootnote\footnote%
\def\footnote{\protect\rmarkdownfootnote}
\IfFileExists{upquote.sty}{\usepackage{upquote}}{}

%%% Include extra packages specified by user

%%% Hard setting column skips for reports - this ensures greater consistency and control over the length settings in the document.
%% page layout
%% paragraphs
\setlength{\baselineskip}{12pt plus 0pt minus 0pt}
\setlength{\parskip}{12pt plus 0pt minus 0pt}
\setlength{\parindent}{0pt plus 0pt minus 0pt}
%% floats
\setlength{\floatsep}{12pt plus 0 pt minus 0pt}
\setlength{\textfloatsep}{20pt plus 0pt minus 0pt}
\setlength{\intextsep}{14pt plus 0pt minus 0pt}
\setlength{\dbltextfloatsep}{20pt plus 0pt minus 0pt}
\setlength{\dblfloatsep}{14pt plus 0pt minus 0pt}
%% maths
\setlength{\abovedisplayskip}{12pt plus 0pt minus 0pt}
\setlength{\belowdisplayskip}{12pt plus 0pt minus 0pt}
%% lists
\setlength{\topsep}{10pt plus 0pt minus 0pt}
\setlength{\partopsep}{3pt plus 0pt minus 0pt}
\setlength{\itemsep}{5pt plus 0pt minus 0pt}
\setlength{\labelsep}{8mm plus 0mm minus 0mm}
\setlength{\parsep}{\the\parskip}
\setlength{\listparindent}{\the\parindent}
%% verbatim
\setlength{\fboxsep}{5pt plus 0pt minus 0pt}



\begin{document}



\begin{frontmatter}  %

\title{Volatility Spillovers from US to SA Markets}

% Set to FALSE if wanting to remove title (for submission)




\author[Add1]{Ruan Geldenhuys}
\ead{22550801@sun.ac.za}

\author[Add1,Add2]{Nico Katzke}
\ead{nfkatzke@gmail.com}




\address[Add1]{Stellenbosch University, Stellenbosch, South Africa}
\address[Add2]{Stellenbosch University, Stellenbosch, South Africa}


\begin{abstract}
\small{
I investigate the relationship between the voaltilities of
S\textbackslash\&P 500 and the JSE Top 40. The purpose of this study is
to investigate if this relationship changes in any significant way
during the two biggest crisis periods in the last two decades, namely
the Global Financial Crisis and Covid-19. I first do a stratification
analysis which reveals significant evidence of these two indices sharing
periods of high volatility. I then fit multiple multivariate GARCH
models to further investigate the volatility relationship and
find\ldots{}
}
\end{abstract}

\vspace{1cm}


\begin{keyword}
\footnotesize{
Multivariate GARCH \sep Spillovers \\
\vspace{0.3cm}
}
\end{keyword}



\vspace{0.5cm}

\end{frontmatter}

\setcounter{footnote}{0}



%________________________
% Header and Footers
%%%%%%%%%%%%%%%%%%%%%%%%%%%%%%%%%
\pagestyle{fancy}
\chead{}
\rhead{}
\lfoot{}
\rfoot{\footnotesize Page \thepage}
\lhead{}
%\rfoot{\footnotesize Page \thepage } % "e.g. Page 2"
\cfoot{}

%\setlength\headheight{30pt}
%%%%%%%%%%%%%%%%%%%%%%%%%%%%%%%%%
%________________________

\headsep 35pt % So that header does not go over title




\hypertarget{introduction}{%
\section{\texorpdfstring{Introduction
\label{Introduction}}{Introduction }}\label{introduction}}

\hypertarget{data-and-methodology}{%
\section{Data and Methodology}\label{data-and-methodology}}

\hypertarget{data}{%
\subsection{Data}\label{data}}

Three return series are used in the analysis that follows. These are the
monthly returns for the S\textbackslash\&P 500 and the JSE Top 40, as
well as thhe ZAR/USD exchange rate. The exchange rate is represented as
the amount of Rands required to buy one US Dollar. Since the series is
represented as a growth rate, a postive growth rate represents a
depreciation of the Rand, and conversely, an appreciation of the Dollar.
The returns for these 3 series' are visualised below in Figures
\ref{Figure1} to \ref{Figure3}.

\begin{figure}[H]

{\centering \includegraphics{Template_files/figure-latex/Figure1-1} 

}

\caption{S\&P 500 Returns \label{Figure1}}\label{fig:Figure1}
\end{figure}

\begin{figure}[H]

{\centering \includegraphics{Template_files/figure-latex/Figure2-1} 

}

\caption{JSE Top 40 Returns \label{Figure2}}\label{fig:Figure2}
\end{figure}

Not much information can be revealed through simply observing the
returns over time. However, when investigating the squared returns as a
measure of volatility, it is clear to see that the JSE Top 40 is
substantially more volatile than the S\textbackslash\&P 500. This result
is reinforced by Table \ref{tab1}, where the JSE showcases a standard
deviation considerably higher than that of the S\textbackslash\&P.
Interestingly, the JSE Top 40 showcases higher average monthly returns,
however that comes at the cost of the increased volatility as described
above. Lastly, as shown in Table \ref{tab1} the S\textbackslash\&P
experienced the largest draw down, while the JSE experienced the largest
uptick.

\begin{figure}[H]

{\centering \includegraphics{Template_files/figure-latex/Figure3-1} 

}

\caption{ZAR/USD Returns \label{Figure3}}\label{fig:Figure3}
\end{figure}

\begin{table}[H]
\centering
\caption{Summary Statistics \label{tab1}} 
\begin{tabular}{lrrr}
  \hline
 & S\&P 500 & JSE Top 40 & ZAR/USD \\ 
  \hline
Mean & 0.0066 & 0.0121 & 0.0040 \\ 
  Median & 0.0124 & 0.0115 & 0.0018 \\ 
  Std. Dev. & 0.0445 & 0.0502 & 0.0485 \\ 
  Kurtosis & 3.7769 & 3.3257 & 4.5258 \\ 
  Skewness & -0.5048 & 0.0359 & 0.2933 \\ 
  Minimum & -0.1680 & -0.1427 & -0.1868 \\ 
  Maximum & 0.1282 & 0.1467 & 0.1875 \\ 
   \hline
\end{tabular}
\end{table}

\hypertarget{methodology}{%
\subsection{Methodology}\label{methodology}}

\hypertarget{results}{%
\section{Results}\label{results}}

\hypertarget{dcc}{%
\subsection{DCC}\label{dcc}}

\hypertarget{go-garch}{%
\subsection{Go-GARCH}\label{go-garch}}

\hypertarget{bekk-garch}{%
\subsection{BEKK-GARCH}\label{bekk-garch}}

\hypertarget{conclusion}{%
\section{Conclusion}\label{conclusion}}

I hope you find this template useful. Remember, stackoverflow is your
friend - use it to find answers to questions. Feel free to write me a
mail if you have any questions regarding the use of this package. To
cite this package, simply type citation(``Texevier'') in Rstudio to get
the citation for Katzke (\protect\hyperlink{ref-Texevier}{2017}) (Note
that uncited references in your bibtex file will not be included in
References).

\newpage

\hypertarget{references}{%
\section*{References}\label{references}}
\addcontentsline{toc}{section}{References}

\hypertarget{refs}{}
\begin{CSLReferences}{1}{0}
\leavevmode\vadjust pre{\hypertarget{ref-Texevier}{}}%
Katzke, N.F. 2017. \emph{{Texevier}: {P}ackage to create elsevier
templates for rmarkdown}. Stellenbosch, South Africa: Bureau for
Economic Research.

\end{CSLReferences}

\hypertarget{appendix}{%
\section*{Appendix}\label{appendix}}
\addcontentsline{toc}{section}{Appendix}

\hypertarget{appendix-a}{%
\subsection*{Appendix A}\label{appendix-a}}
\addcontentsline{toc}{subsection}{Appendix A}

Some appendix information here

\hypertarget{appendix-b}{%
\subsection*{Appendix B}\label{appendix-b}}
\addcontentsline{toc}{subsection}{Appendix B}

\bibliography{Tex/ref}





\end{document}
