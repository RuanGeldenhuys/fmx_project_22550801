\documentclass[11pt,preprint, authoryear]{elsarticle}

\usepackage{lmodern}
%%%% My spacing
\usepackage{setspace}
\setstretch{1.2}
\DeclareMathSizes{12}{14}{10}{10}

% Wrap around which gives all figures included the [H] command, or places it "here". This can be tedious to code in Rmarkdown.
\usepackage{float}
\let\origfigure\figure
\let\endorigfigure\endfigure
\renewenvironment{figure}[1][2] {
    \expandafter\origfigure\expandafter[H]
} {
    \endorigfigure
}

\let\origtable\table
\let\endorigtable\endtable
\renewenvironment{table}[1][2] {
    \expandafter\origtable\expandafter[H]
} {
    \endorigtable
}


\usepackage{ifxetex,ifluatex}
\usepackage{fixltx2e} % provides \textsubscript
\ifnum 0\ifxetex 1\fi\ifluatex 1\fi=0 % if pdftex
  \usepackage[T1]{fontenc}
  \usepackage[utf8]{inputenc}
\else % if luatex or xelatex
  \ifxetex
    \usepackage{mathspec}
    \usepackage{xltxtra,xunicode}
  \else
    \usepackage{fontspec}
  \fi
  \defaultfontfeatures{Mapping=tex-text,Scale=MatchLowercase}
  \newcommand{\euro}{€}
\fi

\usepackage{amssymb, amsmath, amsthm, amsfonts}

\def\bibsection{\section*{References}} %%% Make "References" appear before bibliography


\usepackage[round]{natbib}

\usepackage{longtable}
\usepackage[margin=2.3cm,bottom=2cm,top=2.5cm, includefoot]{geometry}
\usepackage{fancyhdr}
\usepackage[bottom, hang, flushmargin]{footmisc}
\usepackage{graphicx}
\numberwithin{equation}{section}
\numberwithin{figure}{section}
\numberwithin{table}{section}
\setlength{\parindent}{0cm}
\setlength{\parskip}{1.3ex plus 0.5ex minus 0.3ex}
\usepackage{textcomp}
\renewcommand{\headrulewidth}{0.2pt}
\renewcommand{\footrulewidth}{0.3pt}

\usepackage{array}
\newcolumntype{x}[1]{>{\centering\arraybackslash\hspace{0pt}}p{#1}}

%%%%  Remove the "preprint submitted to" part. Don't worry about this either, it just looks better without it:
\makeatletter
\def\ps@pprintTitle{%
  \let\@oddhead\@empty
  \let\@evenhead\@empty
  \let\@oddfoot\@empty
  \let\@evenfoot\@oddfoot
}
\makeatother

 \def\tightlist{} % This allows for subbullets!

\usepackage{hyperref}
\hypersetup{breaklinks=true,
            bookmarks=true,
            colorlinks=true,
            citecolor=blue,
            urlcolor=blue,
            linkcolor=blue,
            pdfborder={0 0 0}}


% The following packages allow huxtable to work:
\usepackage{siunitx}
\usepackage{multirow}
\usepackage{hhline}
\usepackage{calc}
\usepackage{tabularx}
\usepackage{booktabs}
\usepackage{caption}


\newenvironment{columns}[1][]{}{}

\newenvironment{column}[1]{\begin{minipage}{#1}\ignorespaces}{%
\end{minipage}
\ifhmode\unskip\fi
\aftergroup\useignorespacesandallpars}

\def\useignorespacesandallpars#1\ignorespaces\fi{%
#1\fi\ignorespacesandallpars}

\makeatletter
\def\ignorespacesandallpars{%
  \@ifnextchar\par
    {\expandafter\ignorespacesandallpars\@gobble}%
    {}%
}
\makeatother

\newenvironment{CSLReferences}[2]{%
}

\urlstyle{same}  % don't use monospace font for urls
\setlength{\parindent}{0pt}
\setlength{\parskip}{6pt plus 2pt minus 1pt}
\setlength{\emergencystretch}{3em}  % prevent overfull lines
\setcounter{secnumdepth}{5}

%%% Use protect on footnotes to avoid problems with footnotes in titles
\let\rmarkdownfootnote\footnote%
\def\footnote{\protect\rmarkdownfootnote}
\IfFileExists{upquote.sty}{\usepackage{upquote}}{}

%%% Include extra packages specified by user

%%% Hard setting column skips for reports - this ensures greater consistency and control over the length settings in the document.
%% page layout
%% paragraphs
\setlength{\baselineskip}{12pt plus 0pt minus 0pt}
\setlength{\parskip}{12pt plus 0pt minus 0pt}
\setlength{\parindent}{0pt plus 0pt minus 0pt}
%% floats
\setlength{\floatsep}{12pt plus 0 pt minus 0pt}
\setlength{\textfloatsep}{20pt plus 0pt minus 0pt}
\setlength{\intextsep}{14pt plus 0pt minus 0pt}
\setlength{\dbltextfloatsep}{20pt plus 0pt minus 0pt}
\setlength{\dblfloatsep}{14pt plus 0pt minus 0pt}
%% maths
\setlength{\abovedisplayskip}{12pt plus 0pt minus 0pt}
\setlength{\belowdisplayskip}{12pt plus 0pt minus 0pt}
%% lists
\setlength{\topsep}{10pt plus 0pt minus 0pt}
\setlength{\partopsep}{3pt plus 0pt minus 0pt}
\setlength{\itemsep}{5pt plus 0pt minus 0pt}
\setlength{\labelsep}{8mm plus 0mm minus 0mm}
\setlength{\parsep}{\the\parskip}
\setlength{\listparindent}{\the\parindent}
%% verbatim
\setlength{\fboxsep}{5pt plus 0pt minus 0pt}



\begin{document}



\begin{frontmatter}  %

\title{Volatility Spillovers from US to SA Markets}

% Set to FALSE if wanting to remove title (for submission)




\author[Add1]{Ruan Geldenhuys}
\ead{22550801@sun.ac.za}

\author[Add1,Add2]{Nico Katzke}
\ead{nfkatzke@gmail.com}




\address[Add1]{Stellenbosch University, Stellenbosch, South Africa}
\address[Add2]{Stellenbosch University, Stellenbosch, South Africa}


\begin{abstract}
\small{
I investigate the relationship between the voaltilities of S\&P 500 and
the JSE Top 40. The purpose of this study is to investigate if this
relationship changes in any significant way during the two biggest
crisis periods in the last two decades, namely the Global Financial
Crisis and Covid-19. I first do a stratification analysis which reveals
significant evidence of these two indices sharing periods of high
volatility. I then fit multiple multivariate GARCH models to further
investigate the volatility relationship and find\ldots{}
}
\end{abstract}

\vspace{1cm}


\begin{keyword}
\footnotesize{
Multivariate GARCH \sep Spillovers \\
\vspace{0.3cm}
}
\end{keyword}



\vspace{0.5cm}

\end{frontmatter}

\setcounter{footnote}{0}



%________________________
% Header and Footers
%%%%%%%%%%%%%%%%%%%%%%%%%%%%%%%%%
\pagestyle{fancy}
\chead{}
\rhead{}
\lfoot{}
\rfoot{\footnotesize Page \thepage}
\lhead{}
%\rfoot{\footnotesize Page \thepage } % "e.g. Page 2"
\cfoot{}

%\setlength\headheight{30pt}
%%%%%%%%%%%%%%%%%%%%%%%%%%%%%%%%%
%________________________

\headsep 35pt % So that header does not go over title




\hypertarget{introduction}{%
\section{\texorpdfstring{Introduction
\label{Introduction}}{Introduction }}\label{introduction}}

\hypertarget{data}{%
\section{Data}\label{data}}

Three return series are used in the analysis that follows. These are the
monthly returns for the S\&P 500 and the JSE Top 40, as well as thhe
ZAR/USD exchange rate. The exchange rate is represented as the amount of
Rands required to buy one US Dollar. Since the series is represented as
a growth rate, a postive growth rate represents a depreciation of the
Rand, and conversely, an appreciation of the Dollar. The inclusion of an
exchange rate serves primarily as a control variable and as such
analysis regarding the exchange rate is kept to a minimum in the final
analysis. The returns for these 3 series' are visualised below in
Figures \ref{Figure1} to \ref{Figure3}.

\begin{figure}[H]

{\centering \includegraphics{Template_files/figure-latex/Figure1-1} 

}

\caption{S\&P 500 Returns \label{Figure1}}\label{fig:Figure1}
\end{figure}

\begin{figure}[H]

{\centering \includegraphics{Template_files/figure-latex/Figure2-1} 

}

\caption{JSE Top 40 Returns \label{Figure2}}\label{fig:Figure2}
\end{figure}

Not much information can be revealed through simply observing the
returns over time. However, when investigating the squared returns as a
measure of volatility, it is clear to see that the JSE Top 40 is
substantially more volatile than the S\&P 500. This result is reinforced
by Table \ref{tab1}, where the JSE showcases a standard deviation
considerably higher than that of the S\&P. Interestingly, the JSE Top 40
showcases higher average monthly returns, however that comes at the cost
of the increased volatility as described above. Lastly, as shown in
Table \ref{tab1} the S\&P experienced the largest draw down, while the
JSE experienced the largest uptick.

\begin{figure}[H]

{\centering \includegraphics{Template_files/figure-latex/Figure3-1} 

}

\caption{ZAR/USD Returns \label{Figure3}}\label{fig:Figure3}
\end{figure}

\begin{table}[H]
\centering
\caption{Summary Statistics \label{tab1}} 
\begin{tabular}{lrrr}
  \hline
 & S\&P 500 & JSE Top 40 & ZAR/USD \\ 
  \hline
Mean & 0.0066 & 0.0121 & 0.0040 \\ 
  Median & 0.0124 & 0.0115 & 0.0018 \\ 
  Std. Dev. & 0.0445 & 0.0502 & 0.0485 \\ 
  Kurtosis & 3.7769 & 3.3257 & 4.5258 \\ 
  Skewness & -0.5048 & 0.0359 & 0.2933 \\ 
  Minimum & -0.1680 & -0.1427 & -0.1868 \\ 
  Maximum & 0.1282 & 0.1467 & 0.1875 \\ 
   \hline
\end{tabular}
\end{table}

Before GARCH models can be fitted, ARCH tests need to be conducted in
order to see if controlling for conditional heteroskedasticity is
appropriate. I employ two tests. First, a univariate Ljung-Box test is
conducted on each series. Practically, to test for ARCH effects a simple
AR(1) model is fitted for each series and then Ljung-Box tests are done
on the residuals of each AR(1). Next, multivariate Portmanteau tests are
conducted to incorporate all variables simultaneously. As outlined by
Tsay (\protect\hyperlink{ref-Tsay2014}{2014}), 3 tests are tun. The
results can found in the tables below

\begin{table}[H]
\centering
\caption{Ljung-Box Tests \label{tab2}} 
\begin{tabular}{lrrr}
  \hline
Series & TestStatistic & PValue & LagOrder \\ 
  \hline
SP500 & 66.5254 & 0.0000 & 12 \\ 
  JSE40 & 65.7493 & 0.0000 & 12 \\ 
  ZARUSD & 16.0904 & 0.1378 & 12 \\ 
   \hline
\end{tabular}
\end{table}

\begin{table}[H]
\centering
\caption{MV Portmanteau Tests \label{tab3}} 
\begin{tabular}{lrr}
  \hline
 & Test Statistic & p-value \\ 
  \hline
Q(m) of squared series(LM test) & 81.3244 & 0.0001 \\ 
  Rank-based Test & 92.5271 & 0.0001 \\ 
  Q\_k(m) of squared series: & 165.2131 & 0.0001 \\ 
   \hline
\end{tabular}
\end{table}

As can be seen in Table \ref{tab2}, for the S\&P 500 and the JSE Top 40,
the p-values are functionally zero. This means that we can reject the
null of no ARCH effects for these series. The same does not hold for
ZAR/USD exchange rate, which has a p-value greater than the critical
level of 0.05. However when conducting multivariate tests, all tests
report p-values that are functionally zero. As such the analysis
continues with the assumption that ARCH effects are present within the
data. This serves as motivation for the use of GARCH models in this
essay.

\hypertarget{methodology}{%
\section{Methodology}\label{methodology}}

I first perform stratification analysis on all three series to determine
if periods of high or low volatility are shared across the S\&P, the JSE
and the ZAR/USD exchange rate. I then fit multiple univariate GARCH
models on all three variables to determine an appropriate specification
for the multivariate models to come. I then fit three multivariate GARCH
models, namely a DCC model, a Go-GARCH model, and a BEKK-GARCH model.
Formal definitions and explanations of these models follow below.

\hypertarget{dcc-garch}{%
\subsection{DCC GARCH}\label{dcc-garch}}

Dynamic Conditional Correlation (DCC) models, developed by Engle
(\protect\hyperlink{ref-Engle2002}{2002}), are a class of multivariate
GARCH models that allow for time varying correlation between variables.
This is especially useful to study how specific time periods, like
crises, affect the relationship between different stock indices and
financial variables. Consider the following GARCH(1,1) model:

\begin{equation}
R_{it} = \mu_i + \epsilon_{it}, \quad \epsilon_{it} = \sigma_{it}z_{it}, \quad z_{it} \sim N(0,1) \label{eq1}
\end{equation}

In the equation above, \(R_t = (R_{1t}, R_{2t}, ... R_{nt})\), for \(n\)
assets/variables, is a vector of asset returns at time \(t\). Here
\(\mu_i\) is the mean return, \(\epsilon_{it}\) is the residual,
\(\sigma_{it}^2\) is the conditional variance and \(z_{it}\) is the
standard normal innovation. The conditional variance \(\sigma_{it}^2\)
is modeled as:

\begin{equation}
\sigma_{it}^2 = \alpha_0 + \alpha_1 \epsilon_{it-1}^2 + \beta_1 \sigma_{it-1}^2 \label{eq2}
\end{equation}

To specify the DCC model, the standardized residuals are defined as
\(\tilde{\epsilon_t} = \left( \epsilon_{1t} / \sigma_{1t}, \ldots, \epsilon_{nt} / \sigma_{nt} \right)'\).
The correlation matrix of \(\tilde{\epsilon_t}\), denoted by \(Q_t\),
evolves over time as:

\begin{equation}
Q_t = \bar{Q}(1 - a - b) + a \tilde{\epsilon}_{t-1}  \tilde{\epsilon}'_{t-1} + b Q_{t-1} \label{eq3}
\end{equation}

where \(\bar{Q}\) is the unconditional correlation matrix of
\(\tilde{\epsilon_t}\). Parameters, \(a\) and \(b\) are positive and
adhere to \(a + b < 1\) to ensure stationarity. To obtain the dynamic
conditional correlation, the elements of \(Q_t\) are standardized.

In order to estimate a DCC GARCH model, two steps are followed. First,
to obtain \(\sigma_{it}\) and \(\tilde{\epsilon}_t\), a univariate GARCH
is fitted for each return series. Then, secondly, the DCC parameters
\(a\) and \(b\) are estimated using a likelihood function derived from
the conditional multivariate distribution of \(\tilde{\epsilon}_t\).

\hypertarget{go-garch}{%
\subsection{GO-GARCH}\label{go-garch}}

Generalized Orthogonalized (GO) GARCH models are another class of
multivariate GARCH models. Developed by Van der Weide
(\protect\hyperlink{ref-VanDerWeide2002}{2002}), the model is based on
the assumption that asset returns can be decomposed into orthogonal
components, thus simplifying the modeling of their covariance structure.
Note that GO-GARCH models can become computationally intensive quickly,
as the number of variables in the model increases. Once again consider a
a set of \(n\) asset returns, \(R_t = (R_{1t}, R_{2t}, ... R_{nt})\).
These returns can be expressed as a linear combination of its orthogonal
components:

\begin{equation}
R_t = B_t F_t \label{eq4}
\end{equation}

Here \(B_t\) is a time-varying \(n \times n\) matrix of loadings and
\(F_t\) are the orthogonal components,
\(F_t = (F_{1t}, F_{2t}, ... F_{nt})'\), that are assumed to follow a
univariate GARCH process:

\begin{equation}
F_{it} = \sigma_{it}z_{it} \quad z_{it} \sim N(0,1)\label{eq5}
\end{equation}

where \(\sigma^2_{it}\) is the conditional variance of \(F_{it}\). In
order to estimate the model, the loading matrix \(B_t\) is estimated
based on the observed correlation of the series', while the volatilities
of the orthogonal components, \(\sigma^2_{it}\), are estimated using
standard GARCH procedures.

\hypertarget{bekk-garch}{%
\subsection{BEKK-GARCH}\label{bekk-garch}}

The BEKK-GARCH models, initially developed by Engle \& Kroner
(\protect\hyperlink{ref-Bekk1995}{1995}), are designed specifically to
study spillovers between series. The model's ability to capture dynamic
covariances and correlations makes it particularly useful for analyzing
the interdependencies in financial markets, especially in the context of
crises periods. As before consider \(n\) assets with returns,
\(R_t = (R_{1t}, R_{2t}, ... R_{nt})\). For a GARCH(1,1) these returns
are given by:

\begin{equation}
R_t = \mu + \epsilon_t, \quad \epsilon_t = H_t^{1/2} z_t, \quad z_t \sim N(0, I) \label{eq6}
\end{equation}

In Equation \ref{eq6}, as before \(\mu\) is the vector of mean returns
and \(\epsilon_t\) is the vector of residuals. Now, \(H_t\) is the
conditional covariance matrix and \(z_t\) is a vector of i.i.d. standard
normal innovations. The conditional covariance matrix \(H_t\) is modeled
as:

\begin{equation}
H_t = C + A \epsilon_{t-1} \epsilon'_{t-1} A' + B H_{t-1} B' \label{eq7}
\end{equation}

where \(C\), \(A\) and \(B\) are coefficient matrices. Notably, \(C\) is
a triangular matrix with positive diagonal elements, ensuring that it is
positive definite. This attribute removes the need for additional
constraints. Lastly, estimation is done via maximum likelihood.

\hypertarget{results}{%
\section{Results}\label{results}}

\hypertarget{stratification}{%
\subsection{Stratification}\label{stratification}}

\hypertarget{dcc}{%
\subsection{DCC}\label{dcc}}

\hypertarget{go-garch-1}{%
\subsection{Go-GARCH}\label{go-garch-1}}

\hypertarget{bekk-garch-1}{%
\subsection{BEKK-GARCH}\label{bekk-garch-1}}

\hypertarget{conclusion}{%
\section{Conclusion}\label{conclusion}}

I hope you find this template useful. Remember, stackoverflow is your
friend - use it to find answers to questions. Feel free to write me a
mail if you have any questions regarding the use of this package. To
cite this package, simply type citation(``Texevier'') in Rstudio to get
the citation for (\protect\hyperlink{ref-Texevier}{\textbf{Texevier?}})
(Note that uncited references in your bibtex file will not be included
in References).

\newpage

\hypertarget{references}{%
\section*{References}\label{references}}
\addcontentsline{toc}{section}{References}

\hypertarget{refs}{}
\begin{CSLReferences}{1}{0}
\leavevmode\vadjust pre{\hypertarget{ref-Engle2002}{}}%
Engle, R. 2002. Dynamic conditional correlation: A simple class of
multivariate generalized autoregressive conditional heteroskedasticity
models. \emph{Journal of Business \& Economic Statistics}.
20(3):339--350.

\leavevmode\vadjust pre{\hypertarget{ref-Bekk1995}{}}%
Engle, R.F. \& Kroner, K.F. 1995. Multivariate simultaneous generalized
arch. \emph{Econometric Theory}. 11(1):122--150.

\leavevmode\vadjust pre{\hypertarget{ref-Tsay2014}{}}%
Tsay, R.S. 2014. \emph{An introduction to analysis of financial data
with r}. John Wiley \& Sons.

\leavevmode\vadjust pre{\hypertarget{ref-VanDerWeide2002}{}}%
Van der Weide, R. 2002. GO-garch: A multivariate generalized orthogonal
garch model. \emph{Journal of Applied Econometrics}. 17(5):549--564.

\end{CSLReferences}

\hypertarget{appendix}{%
\section*{Appendix}\label{appendix}}
\addcontentsline{toc}{section}{Appendix}

\hypertarget{appendix-a}{%
\subsection*{Appendix A}\label{appendix-a}}
\addcontentsline{toc}{subsection}{Appendix A}

Some appendix information here

\hypertarget{appendix-b}{%
\subsection*{Appendix B}\label{appendix-b}}
\addcontentsline{toc}{subsection}{Appendix B}

\bibliography{Tex/ref}





\end{document}
