\documentclass[11pt,preprint, authoryear]{elsarticle}

\usepackage{lmodern}
%%%% My spacing
\usepackage{setspace}
\setstretch{1.2}
\DeclareMathSizes{12}{14}{10}{10}

% Wrap around which gives all figures included the [H] command, or places it "here". This can be tedious to code in Rmarkdown.
\usepackage{float}
\let\origfigure\figure
\let\endorigfigure\endfigure
\renewenvironment{figure}[1][2] {
    \expandafter\origfigure\expandafter[H]
} {
    \endorigfigure
}

\let\origtable\table
\let\endorigtable\endtable
\renewenvironment{table}[1][2] {
    \expandafter\origtable\expandafter[H]
} {
    \endorigtable
}


\usepackage{ifxetex,ifluatex}
\usepackage{fixltx2e} % provides \textsubscript
\ifnum 0\ifxetex 1\fi\ifluatex 1\fi=0 % if pdftex
  \usepackage[T1]{fontenc}
  \usepackage[utf8]{inputenc}
\else % if luatex or xelatex
  \ifxetex
    \usepackage{mathspec}
    \usepackage{xltxtra,xunicode}
  \else
    \usepackage{fontspec}
  \fi
  \defaultfontfeatures{Mapping=tex-text,Scale=MatchLowercase}
  \newcommand{\euro}{€}
\fi

\usepackage{amssymb, amsmath, amsthm, amsfonts}

\def\bibsection{\section*{References}} %%% Make "References" appear before bibliography


\usepackage[round]{natbib}

\usepackage{longtable}
\usepackage[margin=2.3cm,bottom=2cm,top=2.5cm, includefoot]{geometry}
\usepackage{fancyhdr}
\usepackage[bottom, hang, flushmargin]{footmisc}
\usepackage{graphicx}
\numberwithin{equation}{section}
\numberwithin{figure}{section}
\numberwithin{table}{section}
\setlength{\parindent}{0cm}
\setlength{\parskip}{1.3ex plus 0.5ex minus 0.3ex}
\usepackage{textcomp}
\renewcommand{\headrulewidth}{0.2pt}
\renewcommand{\footrulewidth}{0.3pt}

\usepackage{array}
\newcolumntype{x}[1]{>{\centering\arraybackslash\hspace{0pt}}p{#1}}

%%%%  Remove the "preprint submitted to" part. Don't worry about this either, it just looks better without it:
\makeatletter
\def\ps@pprintTitle{%
  \let\@oddhead\@empty
  \let\@evenhead\@empty
  \let\@oddfoot\@empty
  \let\@evenfoot\@oddfoot
}
\makeatother

 \def\tightlist{} % This allows for subbullets!

\usepackage{hyperref}
\hypersetup{breaklinks=true,
            bookmarks=true,
            colorlinks=true,
            citecolor=blue,
            urlcolor=blue,
            linkcolor=blue,
            pdfborder={0 0 0}}


% The following packages allow huxtable to work:
\usepackage{siunitx}
\usepackage{multirow}
\usepackage{hhline}
\usepackage{calc}
\usepackage{tabularx}
\usepackage{booktabs}
\usepackage{caption}


\newenvironment{columns}[1][]{}{}

\newenvironment{column}[1]{\begin{minipage}{#1}\ignorespaces}{%
\end{minipage}
\ifhmode\unskip\fi
\aftergroup\useignorespacesandallpars}

\def\useignorespacesandallpars#1\ignorespaces\fi{%
#1\fi\ignorespacesandallpars}

\makeatletter
\def\ignorespacesandallpars{%
  \@ifnextchar\par
    {\expandafter\ignorespacesandallpars\@gobble}%
    {}%
}
\makeatother

\newenvironment{CSLReferences}[2]{%
}

\urlstyle{same}  % don't use monospace font for urls
\setlength{\parindent}{0pt}
\setlength{\parskip}{6pt plus 2pt minus 1pt}
\setlength{\emergencystretch}{3em}  % prevent overfull lines
\setcounter{secnumdepth}{5}

%%% Use protect on footnotes to avoid problems with footnotes in titles
\let\rmarkdownfootnote\footnote%
\def\footnote{\protect\rmarkdownfootnote}
\IfFileExists{upquote.sty}{\usepackage{upquote}}{}

%%% Include extra packages specified by user

%%% Hard setting column skips for reports - this ensures greater consistency and control over the length settings in the document.
%% page layout
%% paragraphs
\setlength{\baselineskip}{12pt plus 0pt minus 0pt}
\setlength{\parskip}{12pt plus 0pt minus 0pt}
\setlength{\parindent}{0pt plus 0pt minus 0pt}
%% floats
\setlength{\floatsep}{12pt plus 0 pt minus 0pt}
\setlength{\textfloatsep}{20pt plus 0pt minus 0pt}
\setlength{\intextsep}{14pt plus 0pt minus 0pt}
\setlength{\dbltextfloatsep}{20pt plus 0pt minus 0pt}
\setlength{\dblfloatsep}{14pt plus 0pt minus 0pt}
%% maths
\setlength{\abovedisplayskip}{12pt plus 0pt minus 0pt}
\setlength{\belowdisplayskip}{12pt plus 0pt minus 0pt}
%% lists
\setlength{\topsep}{10pt plus 0pt minus 0pt}
\setlength{\partopsep}{3pt plus 0pt minus 0pt}
\setlength{\itemsep}{5pt plus 0pt minus 0pt}
\setlength{\labelsep}{8mm plus 0mm minus 0mm}
\setlength{\parsep}{\the\parskip}
\setlength{\listparindent}{\the\parindent}
%% verbatim
\setlength{\fboxsep}{5pt plus 0pt minus 0pt}



\begin{document}



\begin{frontmatter}  %

\title{Volatility Spillovers from US to SA Markets}

% Set to FALSE if wanting to remove title (for submission)




\author[Add1]{Ruan Geldenhuys}
\ead{22550801@sun.ac.za}

\author[Add1,Add2]{Nico Katzke}
\ead{nfkatzke@gmail.com}




\address[Add1]{Stellenbosch University, Stellenbosch, South Africa}
\address[Add2]{Stellenbosch University, Stellenbosch, South Africa}


\begin{abstract}
\small{
I investigate the relationship between the S\&P 500 and the JSE Top 40,
through various volatility modeling techniques. The purpose of this
study is to investigate if this relationship changes in any significant
way during the two biggest crisis periods in the last two decades,
namely the Global Financial Crisis and Covid-19. I first do a
stratification analysis which reveals significant evidence of these two
indices sharing periods of high volatility. I then fit multiple
multivariate GARCH models to further investigate the relationship and
find conditional correlation to increase during crisis periods. Lastly,
an investigation of volatility linkages reveals no evidence of
volatility spillovers.
}
\end{abstract}

\vspace{1cm}


\begin{keyword}
\footnotesize{
Multivariate GARCH \sep Spillovers \\
\vspace{0.3cm}
}
\end{keyword}



\vspace{0.5cm}

\end{frontmatter}

\setcounter{footnote}{0}



%________________________
% Header and Footers
%%%%%%%%%%%%%%%%%%%%%%%%%%%%%%%%%
\pagestyle{fancy}
\chead{}
\rhead{}
\lfoot{}
\rfoot{\footnotesize Page \thepage}
\lhead{}
%\rfoot{\footnotesize Page \thepage } % "e.g. Page 2"
\cfoot{}

%\setlength\headheight{30pt}
%%%%%%%%%%%%%%%%%%%%%%%%%%%%%%%%%
%________________________

\headsep 35pt % So that header does not go over title




\hypertarget{introduction}{%
\section{\texorpdfstring{Introduction
\label{Introduction}}{Introduction }}\label{introduction}}

In my honours thesis I investigate spillover effects from the S\&P 500
to the JSE Top 40, utilising a Structural VAR. Subsequently, in a time
series econometrics essay I applied similar analysis, this time
estimating a Bayesian VAR. Now, I turn to various volatility modeling
methods in order to uncover new insights into the dynamics between these
stock indices. The US stock market's position as a market leader is a
well established fact within the literature. It therefore becomes
imperative to understand our own local stock market's relationship with
the US, especially in times of crisis.

\hypertarget{data}{%
\section{Data}\label{data}}

Three return series are used in the analysis that follows. These are the
monthly returns for the S\&P 500 and the JSE Top 40, as well as thhe
ZAR/USD exchange rate. The exchange rate is represented as the amount of
Rands required to buy one US Dollar. Since the series is represented as
a growth rate, a postive growth rate represents a depreciation of the
Rand, and conversely, an appreciation of the Dollar. The inclusion of an
exchange rate serves primarily as a control variable and as such
analysis regarding the exchange rate is kept to a minimum in the final
analysis. The returns for these 3 series' are visualised below in
Figures \ref{Figure1} to \ref{Figure3}.

\begin{figure}[H]

{\centering \includegraphics{Template_files/figure-latex/Figure1-1} 

}

\caption{S\&P 500 Returns \label{Figure1}}\label{fig:Figure1}
\end{figure}

\begin{figure}[H]

{\centering \includegraphics{Template_files/figure-latex/Figure2-1} 

}

\caption{JSE Top 40 Returns \label{Figure2}}\label{fig:Figure2}
\end{figure}

Not much information can be revealed through simply observing the
returns over time. However, when investigating the squared returns as a
measure of volatility, it is clear to see that the JSE Top 40 is
substantially more volatile than the S\&P 500. This result is reinforced
by Table \ref{tab1}, where the JSE showcases a standard deviation
considerably higher than that of the S\&P. Interestingly, the JSE Top 40
showcases higher average monthly returns, however that comes at the cost
of the increased volatility as described above. Lastly, as shown in
Table \ref{tab1} the S\&P experienced the largest draw down, while the
JSE experienced the largest uptick.

\begin{figure}[H]

{\centering \includegraphics{Template_files/figure-latex/Figure3-1} 

}

\caption{ZAR/USD Returns \label{Figure3}}\label{fig:Figure3}
\end{figure}

\begin{table}[H]
\centering
\caption{Summary Statistics \label{tab1}} 
\begin{tabular}{lrrr}
  \hline
 & S\&P 500 & JSE Top 40 & ZAR/USD \\ 
  \hline
Mean & 0.0066 & 0.0121 & 0.0040 \\ 
  Median & 0.0124 & 0.0115 & 0.0018 \\ 
  Std. Dev. & 0.0445 & 0.0502 & 0.0485 \\ 
  Kurtosis & 3.7769 & 3.3257 & 4.5258 \\ 
  Skewness & -0.5048 & 0.0359 & 0.2933 \\ 
  Minimum & -0.1680 & -0.1427 & -0.1868 \\ 
  Maximum & 0.1282 & 0.1467 & 0.1875 \\ 
   \hline
\end{tabular}
\end{table}

Before GARCH models can be fitted, ARCH tests need to be conducted in
order to see if controlling for conditional heteroskedasticity is
appropriate. I employ two tests. First, a univariate Ljung-Box test is
conducted on each series. Practically, to test for ARCH effects a simple
AR(1) model is fitted for each series and then Ljung-Box tests are done
on the residuals of each AR(1). Next, multivariate Portmanteau tests are
conducted to incorporate all variables simultaneously. As outlined by
Tsay (\protect\hyperlink{ref-Tsay2014}{2014}), 3 tests are tun. The
results can found in the tables below

\begin{table}[H]
\centering
\caption{Ljung-Box Tests \label{tab2}} 
\begin{tabular}{lrrr}
  \hline
Series & TestStatistic & PValue & LagOrder \\ 
  \hline
SP500 & 66.5254 & 0.0000 & 12 \\ 
  JSE40 & 65.7493 & 0.0000 & 12 \\ 
  ZARUSD & 16.0904 & 0.1378 & 12 \\ 
   \hline
\end{tabular}
\end{table}

\begin{table}[H]
\centering
\caption{MV Portmanteau Tests \label{tab3}} 
\begin{tabular}{lrr}
  \hline
 & Test Statistic & p-value \\ 
  \hline
Q(m) of squared series(LM test) & 81.3244 & 0.0001 \\ 
  Rank-based Test & 92.5271 & 0.0001 \\ 
  Q\_k(m) of squared series: & 165.2131 & 0.0001 \\ 
   \hline
\end{tabular}
\end{table}

As can be seen in Table \ref{tab2}, for the S\&P 500 and the JSE Top 40,
the p-values are functionally zero. This means that we can reject the
null of no ARCH effects for these series. The same does not hold for
ZAR/USD exchange rate, which has a p-value greater than the critical
level of 0.05. However when conducting multivariate tests, all tests
report p-values that are functionally zero. As such the analysis
continues with the assumption that ARCH effects are present within the
data. This serves as motivation for the use of GARCH models in this
essay.

\hypertarget{methodology}{%
\section{Methodology}\label{methodology}}

I first perform stratification analysis on all three series to determine
if periods of high or low volatility are shared across the S\&P, the JSE
and the ZAR/USD exchange rate. I then fit multiple univariate GARCH
models on all three variables to determine an appropriate specification
for the multivariate models to come. I then fit three multivariate GARCH
models, namely a DCC model, a Go-GARCH model, and a BEKK-GARCH model.
Formal definitions and explanations of these models follow below.

\hypertarget{dcc-garch}{%
\subsection{DCC GARCH}\label{dcc-garch}}

Dynamic Conditional Correlation (DCC) models, developed by Engle
(\protect\hyperlink{ref-Engle2002}{2002}), are a class of multivariate
GARCH models that allow for time varying correlation between variables.
This is especially useful to study how specific time periods, like
crises, affect the relationship between different stock indices and
financial variables. Consider the following GARCH(1,1) model:

\begin{equation}
R_{it} = \mu_i + \epsilon_{it}, \quad \epsilon_{it} = \sigma_{it}z_{it}, \quad z_{it} \sim N(0,1) \label{eq1}
\end{equation}

In the equation above, \(R_t = (R_{1t}, R_{2t}, ... R_{nt})\), for \(n\)
assets/variables, is a vector of asset returns at time \(t\). Here
\(\mu_i\) is the mean return, \(\epsilon_{it}\) is the residual,
\(\sigma_{it}^2\) is the conditional variance and \(z_{it}\) is the
standard normal innovation. The conditional variance \(\sigma_{it}^2\)
is modeled as:

\begin{equation}
\sigma_{it}^2 = \alpha_0 + \alpha_1 \epsilon_{it-1}^2 + \beta_1 \sigma_{it-1}^2 \label{eq2}
\end{equation}

To specify the DCC model, the standardized residuals are defined as
\(\tilde{\epsilon_t} = \left( \epsilon_{1t} / \sigma_{1t}, \ldots, \epsilon_{nt} / \sigma_{nt} \right)'\).
The correlation matrix of \(\tilde{\epsilon_t}\), denoted by \(Q_t\),
evolves over time as:

\begin{equation}
Q_t = \bar{Q}(1 - a - b) + a \tilde{\epsilon}_{t-1}  \tilde{\epsilon}'_{t-1} + b Q_{t-1} \label{eq3}
\end{equation}

where \(\bar{Q}\) is the unconditional correlation matrix of
\(\tilde{\epsilon_t}\). Parameters, \(a\) and \(b\) are positive and
adhere to \(a + b < 1\) to ensure stationarity. To obtain the dynamic
conditional correlation, the elements of \(Q_t\) are standardized.

In order to estimate a DCC GARCH model, two steps are followed. First,
to obtain \(\sigma_{it}\) and \(\tilde{\epsilon}_t\), a univariate GARCH
is fitted for each return series. Then, secondly, the DCC parameters
\(a\) and \(b\) are estimated using a likelihood function derived from
the conditional multivariate distribution of \(\tilde{\epsilon}_t\).

\hypertarget{go-garch}{%
\subsection{GO-GARCH}\label{go-garch}}

Generalized Orthogonalized (GO) GARCH models are another class of
multivariate GARCH models. Developed by Van der Weide
(\protect\hyperlink{ref-VanDerWeide2002}{2002}), the model is based on
the assumption that asset returns can be decomposed into orthogonal
components, thus simplifying the modeling of their covariance structure.
Note that GO-GARCH models can become computationally intensive quickly,
as the number of variables in the model increases. Once again consider a
a set of \(n\) asset returns, \(R_t = (R_{1t}, R_{2t}, ... R_{nt})\).
These returns can be expressed as a linear combination of its orthogonal
components:

\begin{equation}
R_t = B_t F_t \label{eq4}
\end{equation}

Here \(B_t\) is a time-varying \(n \times n\) matrix of loadings and
\(F_t\) are the orthogonal components,
\(F_t = (F_{1t}, F_{2t}, ... F_{nt})'\), that are assumed to follow a
univariate GARCH process:

\begin{equation}
F_{it} = \sigma_{it}z_{it} \quad z_{it} \sim N(0,1)\label{eq5}
\end{equation}

where \(\sigma^2_{it}\) is the conditional variance of \(F_{it}\). In
order to estimate the model, the loading matrix \(B_t\) is estimated
based on the observed correlation of the series', while the volatilities
of the orthogonal components, \(\sigma^2_{it}\), are estimated using
standard GARCH procedures.

\hypertarget{bekk-garch}{%
\subsection{BEKK-GARCH}\label{bekk-garch}}

The BEKK-GARCH models, initially developed by Engle \& Kroner
(\protect\hyperlink{ref-Bekk1995}{1995}), are designed specifically to
study spillovers between series. The model's ability to capture dynamic
covariances and correlations makes it particularly useful for analyzing
the interdependencies in financial markets, especially in the context of
crises periods. As before consider \(n\) assets with returns,
\(R_t = (R_{1t}, R_{2t}, ... R_{nt})\). For a GARCH(1,1) these returns
are given by:

\begin{equation}
R_t = \mu + \epsilon_t, \quad \epsilon_t = H_t^{1/2} z_t, \quad z_t \sim N(0, I) \label{eq6}
\end{equation}

In Equation \ref{eq6}, as before \(\mu\) is the vector of mean returns
and \(\epsilon_t\) is the vector of residuals. Now, \(H_t\) is the
conditional covariance matrix and \(z_t\) is a vector of i.i.d. standard
normal innovations. The conditional covariance matrix \(H_t\) is modeled
as:

\begin{equation}
H_t = C + A \epsilon_{t-1} \epsilon'_{t-1} A' + B H_{t-1} B' \label{eq7}
\end{equation}

where \(C\), \(A\) and \(B\) are coefficient matrices. Notably, \(C\) is
a triangular matrix with positive diagonal elements, ensuring that it is
positive definite. This attribute removes the need for additional
constraints. Lastly, estimation is done via maximum likelihood. For
interpretation purposes, the variance of he first asset return can be
written as follows:

\begin{align}
\sigma^2_{1,t} & = C(1,1)^2 + A(1,1)^2\mu^2_{1,t-1} + 2A(1,1)A(2,1)\mu_{1,t-1}\mu_{2,t-1} + A(2,1)^2\mu^2_{2,t-1} \label{eq8}\\
& + B(1,1)^2\sigma^2_{1,t-1} + 2B(1,1)B(2,1)\sigma_{1,t-1} + B(2,1)^2\sigma^2_{2,t-1} 
\end{align}

\hypertarget{results}{%
\section{Results}\label{results}}

Like stated above, I first employ stratification analysis. I then fit
three multivariate GARCH models. The univariate specification these
models are based on are selected by fitting different specifications and
selecting the best one based on various selection criteria. These
results show a gjrGARCH to be the best specification.
\footnote{These results are not reported in this document since they serve little purpose other than model construction. For a full table showing the test results see : \url{https://github.com/RuanGeldenhuys/fmx_project_22550801}}.

\hypertarget{stratification}{%
\subsection{Stratification}\label{stratification}}

Stratification analyses allows for the investigation of a particular
assets volatility during a particular period of volatility of another
asset. While such analysis does not lend itself to causal interpretation
regarding volatility spillovers, it does paint a picture of whether the
indices that are being investigated tend to have periods of high or low
volatility at the same time. In turn, this could point to a direction of
interconnectedness between markets which can then be revealed through
more robust analysis. The stratification analyses follows below.

\begin{table}[H]
\centering
\caption{S\&P 500 High Volatility \label{tab4}} 
\begin{tabular}{lrrlr}
  \hline
Index & SD & Full\_SD & Period & Ratio \\ 
  \hline
JSE40 & 0.23 & 0.17 & High\_Vol SP500 & 1.35 \\ 
  ZARUSD & 0.19 & 0.16 & High\_Vol SP500 & 1.17 \\ 
   \hline
\end{tabular}
\end{table}
\begin{table}[H]
\centering
\caption{S\&P 500 Low Volatility \label{tab5}} 
\begin{tabular}{lrrlr}
  \hline
Index & SD & Full\_SD & Period & Ratio \\ 
  \hline
ZARUSD & 0.16 & 0.16 & Low\_Vol SP500 & 1.03 \\ 
  JSE40 & 0.14 & 0.17 & Low\_Vol SP500 & 0.81 \\ 
   \hline
\end{tabular}
\end{table}

Tables \ref{tab4} and \ref{tab5} showcase the stratification of high and
low volatility of the S\&P 500. The ``SD'' column report volatility
during that particular period while the ``Full\_SD'' column shows the
volatility for the entire sample period. As such a ratio greater than 1
indicates that particular index or series has a higher than usual
volatility in a given period. Analysing periods of high volatility of
the S\&P show both the JSE and the ZAR/USD also have significantly
higher volatility in these periods, indicated by the ratio
\textgreater{} 1. The JSE also reports lower volatility in periods where
the S\&P 500 is less volatile, although the same is not true for the
exchange rate.

\begin{table}[H]
\centering
\caption{JSE Top 40 High Volatility \label{tab6}} 
\begin{tabular}{lrrlr}
  \hline
Index & SD & Full\_SD & Period & Ratio \\ 
  \hline
SP500 & 0.20 & 0.15 & High\_Vol JSE40 & 1.37 \\ 
  ZARUSD & 0.19 & 0.16 & High\_Vol JSE40 & 1.17 \\ 
   \hline
\end{tabular}
\end{table}
\begin{table}[H]
\centering
\caption{JSE Top 40 Low Volatility \label{tab7}} 
\begin{tabular}{lrrlr}
  \hline
Index & SD & Full\_SD & Period & Ratio \\ 
  \hline
ZARUSD & 0.13 & 0.16 & Low\_Vol JSE40 & 0.82 \\ 
  SP500 & 0.10 & 0.15 & Low\_Vol JSE40 & 0.70 \\ 
   \hline
\end{tabular}
\end{table}

Tables \ref{tab6} and \ref{tab7} now report the stratification for the
JSE Top 40. For high volatility periods, the relationship holds with
both the S\&P 500 and the exchange rate showcasing higher than average
volatility. In low volatility periods, again, both the S\&P and the
ZAR/USD report lower than usual volatility. Like this stated earlier,
this does not allow for causal interpretation, but it does point to the
fact that the two indices tend to be in periods of high or low
volatility at the same time.

\begin{table}[H]
\centering
\caption{ZAR/USD High Volatility \label{tab8}} 
\begin{tabular}{lrrlr}
  \hline
Index & SD & Full\_SD & Period & Ratio \\ 
  \hline
JSE40 & 0.18 & 0.17 & High\_Vol ZARUSD & 1.04 \\ 
  SP500 & 0.16 & 0.15 & High\_Vol ZARUSD & 1.07 \\ 
   \hline
\end{tabular}
\end{table}
\begin{table}[H]
\centering
\caption{ZAR/USD Low Volatility \label{tab9}} 
\begin{tabular}{lrrlr}
  \hline
Index & SD & Full\_SD & Period & Ratio \\ 
  \hline
JSE40 & 0.13 & 0.17 & Low\_Vol ZARUSD & 0.79 \\ 
  SP500 & 0.13 & 0.15 & Low\_Vol ZARUSD & 0.88 \\ 
   \hline
\end{tabular}
\end{table}

Lastly, analysing stratification of the ZAR/USD reveals an interesting
result. In periods of high volatility (Table \ref{tab8}), both the JSE
and S\&P also report higher volatility. However this volatility is only
slighter higher, with both indices reporting a standard deviation that
is only 0.01 larger than the full sample. Conversely, in periods of low
volatility (Table \ref{tab9}), these indices show significant lower
volatility as well. This indicates that a highly volatile Rand does not
necessarily mean volatile stock markets, however a low volatile Rand
tends to be associated with low volatility in these indices.

\hypertarget{dcc}{%
\subsection{DCC}\label{dcc}}

\begin{figure}[H]

{\centering \includegraphics{Template_files/figure-latex/Figure4-1} 

}

\caption{DCC GARCH \label{Figure4}}\label{fig:Figure4}
\end{figure}

The dynamic conditional correlation for the JSE Top 40, as reported by
the DCC-GARCH model, are shown in Figure \ref{Figure4}. Intuitively, it
shows a noise reduced correlation between the JSE and other variables in
the system over time. What is immediately clear is that the JSE shares a
significantly higher correlation with the S\&P 500, than it does with
the ZAR/USD exchange rate. In fact, the correlation with the ZAR/USD is
negative for much of the sample period. This result does make sense
since ``increases'' in the exchange rate indicate a depreciation of the
Rand.

Analyzing the crisis periods, particularly in the context of the
correlation between the JSE Top 40 and the S\&P 500, reveals an
interesting result. Note, the GFC period is indicated by the red shaded
area, while Covid-19 is shown in blue. During the GFC the correlation
jumps up initially, then turns sharply downward, before tending upward
for the rest of the crisis. The difference in correlation between the
start of the crises and the maximum correlation is equal to 0.075, while
the difference between the start and end of the crisis is equal to
0.063.

Correlation during Covid-19 behaves differently. Here the model reports
a sharp increase in correlation that fades out as the crisis draws to a
close. As such the maximum correlation appears early in the crisis
period. The difference in correlation between the start of the crisis
period and the maximum is equal to 0.115. The difference between the
start and end of the crisis is smaller at 0.010.

\hypertarget{go-garch-1}{%
\subsection{Go-GARCH}\label{go-garch-1}}

\begin{figure}[H]

{\centering \includegraphics{Template_files/figure-latex/Figure5-1} 

}

\caption{GO-GARCH \label{Figure5}}\label{fig:Figure5}
\end{figure}

The dynamic conditional correlations for the JSE Top 40, as estimated by
the GO-GARCH model is reported in Figure \ref{Figure5}. It is
immediately apparent that the correlation with the exchange rate is much
more volatile than in the DCC model. It is still negative for most of
the sample, however experiences large spikes, turning the relationship
positive. The correlation with the S\&P 500 follows a similar time path
to the DCC model, in crises periods. As such the GO-GARCH serves as a
robustness check for the results found by the DCC model.

A key difference between the model is the fact that the jump in
correlation during crisis periods are more pronounced. The difference in
correlation between the start and maximum of the GFC is equal to 0.362,
while the difference between the first and last correlation is equal to
0.219. For Covid-19, the difference between the first correlation and
maximum correlation is 0.278, and interestingly the difference between
the first and last is negative at -0.011.

\hypertarget{bekk-garch-1}{%
\subsection{BEKK-GARCH}\label{bekk-garch-1}}

The estimated parameters for the BEKK-GARCH model can be found in Table
\ref{tab10} below.

\begin{table}[H]
\centering
\caption{BEKK-GARCH Results \label{tab10}} 
\begin{tabular}{lrrrl}
  \hline
  & Coefficient & Std. Error & TStat & Significance \\ 
  \hline
C(1,1) & 0.0381 & 0.0062 & 6.1298 & *** \\ 
  C(1,2) & 0.0182 & 0.0150 & 1.2184 &  \\ 
  C(1,3) & -0.0143 & 0.0049 & -2.9025 & *** \\ 
  C(2,2) & 0.0150 & 0.0200 & 0.7495 &  \\ 
  C(2,3) & -0.0091 & 0.0175 & -0.5237 &  \\ 
  C(3,3) & 0.0391 & 0.0070 & 5.5738 & *** \\ 
  A(1,1) & 0.0539 & 0.1958 & 0.2753 &  \\ 
  A(1,2) & -0.3134 & 0.2961 & -1.0581 &  \\ 
  A(1,3) & -0.3344 & 0.1863 & -1.7950 & * \\ 
  A(2,1) & -0.3058 & 0.1514 & -2.0200 & ** \\ 
  A(2,2) & 0.1428 & 0.1915 & 0.7455 &  \\ 
  A(2,3) & 0.5942 & 0.1498 & 3.9675 & *** \\ 
  A(3,1) & 0.2485 & 0.0895 & 2.7782 & *** \\ 
  A(3,2) & 0.2343 & 0.1199 & 1.9541 & * \\ 
  A(3,3) & -0.0622 & 0.1223 & -0.5083 &  \\ 
  B(1,1) & -0.0013 & 0.1914 & -0.0070 &  \\ 
  B(1,2) & -0.0216 & 0.5822 & -0.0370 &  \\ 
  B(1,3) & -0.0150 & 0.0758 & -0.1982 &  \\ 
  B(2,1) & 0.0323 & 0.1297 & 0.2488 &  \\ 
  B(2,2) & 0.0814 & 0.3297 & 0.2468 &  \\ 
  B(2,3) & -0.0049 & 0.0646 & -0.0763 &  \\ 
  B(3,1) & -0.2760 & 0.3255 & -0.8479 &  \\ 
  B(3,2) & -0.8576 & 0.2674 & -3.2074 & *** \\ 
  B(3,3) & -0.0662 & 0.1848 & -0.3581 &  \\ 
   \hline
\end{tabular}
\end{table}

Since the main goal of estimating a BEKK-GARCH model, in this essay, is
to measure volatility spillovers, I focus on the B matrix of parameters
as defined in Equation \ref{eq7}. Recall, Equation \ref{eq8} for details
on interpretation. In this model the S\&P 500 is variable 1, the JSE Top
40 is variable 2, and the ZAR/USD exchange rate is variable 3. This
means that coefficient of B(1,2) represents the volatility spillover
from the S\&P 500 to the JSE Top 40.

An interesting result the model produces is that no volatility
spillovers between stock indices are statistically significant. The only
significant volatility spillover appears to be B(3,2), representing a
spillover from the ZAR/USD to the JSE Top 40. This result is peculiar to
say the least. One would expect there to exists some level of volatility
spillovers between stock indices, especially from large markets like the
S\&P 500 to significantly smaller markets like the JSE, as the
literature has shown. Earlier models showed that a relationship between
these indices do in fact exist, however this model was incapable of
revealing such relationship between the volatilities.

\hypertarget{conclusion}{%
\section{Conclusion}\label{conclusion}}

In summary, I find the S\&P 500 and the JSE Top 40 to largely share
periods of high and low volatility. These markets are highly correlated,
as made evident by the DCC and GO-GARCH models. These model further
reveal conditional correlation to have spikes during the Global
Financial Crises and Covid-19. An analysis of volatility spillovers
revealed evidence of spillovers from the ZAR/USD exchange rate, but not
between stock indices. This result is perplexing as literature often
reveals such relationship between the US and emerging market economies.

It is important to note that this analysis only included the largest
stocks for each stock market. As such, no claims can be made regarding
whether the dynamics revealed by these models will hold for lower market
cap stocks. Additionally, no cross-country comparisons have been made. A
study across multiple emerging markets, not only South Africa, is a
likely area of future research. Lastly, if evidence of volatility
spillovers is found an extension to time varying parameters could be
beneficial in order to study the effect of crisis periods in more
detail.

\newpage

\hypertarget{references}{%
\section*{References}\label{references}}
\addcontentsline{toc}{section}{References}

\hypertarget{refs}{}
\begin{CSLReferences}{1}{0}
\leavevmode\vadjust pre{\hypertarget{ref-Engle2002}{}}%
Engle, R. 2002. Dynamic conditional correlation: A simple class of
multivariate generalized autoregressive conditional heteroskedasticity
models. \emph{Journal of Business \& Economic Statistics}.
20(3):339--350.

\leavevmode\vadjust pre{\hypertarget{ref-Bekk1995}{}}%
Engle, R.F. \& Kroner, K.F. 1995. Multivariate simultaneous generalized
arch. \emph{Econometric Theory}. 11(1):122--150.

\leavevmode\vadjust pre{\hypertarget{ref-Tsay2014}{}}%
Tsay, R.S. 2014. \emph{An introduction to analysis of financial data
with r}. John Wiley \& Sons.

\leavevmode\vadjust pre{\hypertarget{ref-VanDerWeide2002}{}}%
Van der Weide, R. 2002. GO-garch: A multivariate generalized orthogonal
garch model. \emph{Journal of Applied Econometrics}. 17(5):549--564.

\end{CSLReferences}

\bibliography{Tex/ref}





\end{document}
